
%% Dokumenteinstellungen %%%%%%%%%%%%%%%%%%%%%%%%%%%%%%%%%%%%
\documentclass[a4paper,oneside,12pt,ngerman]{scrartcl}

%% Deutsche Anpassungen %%%%%%%%%%%%%%%%%%%%%%%%%%%%%%%%%%%%%
\usepackage[ngerman]{babel}
\usepackage[T1]{fontenc}
\usepackage[ansinew]{inputenc}
\usepackage{lmodern} %Type1-Schriftart f�r nicht-englische Texte
\usepackage{booktabs}	% sch�nere tabellen

%% Packages f�r Grafiken & Abbildungen %%%%%%%%%%%%%%%%%%%%%%
\usepackage{graphicx} %%Zum Laden von Grafiken
%\usepackage{subfig} %%Teilabbildungen in einer Abbildung
%\usepackage{tikz} %%Vektorgrafiken aus LaTeX heraus erstellen


%% Packages f�r Formeln %%%%%%%%%%%%%%%%%%%%%%%%%%%%%%%%%%%%%
\usepackage{amsmath}
\usepackage{amsthm}
\usepackage{amsfonts}


%% Andere Packages %%%%%%%%%%%%%%%%%%%%%%%%%%%%%%%%%%%%%%%%%%
%\usepackage{a4wide} %%Kleinere Seitenr�nder = mehr Text pro Zeile.
\usepackage{fancyhdr} %%Fancy Kopf- und Fu�zeilen
%\usepackage{longtable} %%F�r Tabellen, die eine Seite �berschreiten
\usepackage{lastpage}
\usepackage[raggedright]{subfigure}
\usepackage[final]{pdfpages}
\includepdfset{pages=-,noautoscale}

%%%%%%%%%%%%%%%%%%%%%%%%%%%%%%%%%%%%%%%%%%%%%%%%%%%%%%%%%%%%%
%% TODO
%%%%%%%%%%%%%%%%%%%%%%%%%%%%%%%%%%%%%%%%%%%%%%%%%%%%%%%%%%%%%
% 
% 
%%%%%%%%%%%%%%%%%%%%%%%%%%%%%%%%%%%%%%%%%%%%%%%%%%%%%%%%%%%%%



%%%%%%%%%%%%%%%%%%%%%%%%%%%%%%%%%%%%%%%%%%%%%%%%%%%%%%%%%%%%%
%% Optionen / Modifikationen
%%%%%%%%%%%%%%%%%%%%%%%%%%%%%%%%%%%%%%%%%%%%%%%%%%%%%%%%%%%%%
%%%%%%%%%%%%%%%%%%%%%%%%%%%%%%%%%%%%%%%%%%%%%%%%%%%%%%%%%%%%%
%%                                                         %%
%%                     EINSTELLUNGEN                       %%
%%                                                         %%
%%%%%%%%%%%%%%%%%%%%%%%%%%%%%%%%%%%%%%%%%%%%%%%%%%%%%%%%%%%%%

%%%%%%%%%%%%%%%%%%%%%%%%%%%%%%%%%%%%%%%%%%%%%%%%%%%%%%%%%%%%%
%% HYPER REF
%%%%%%%%%%%%%%%%%%%%%%%%%%%%%%%%%%%%%%%%%%%%%%%%%%%%%%%%%%%%%
\usepackage[
hyperindex=true,
colorlinks=true,
linkcolor=black,
citecolor=black,
filecolor=black,
menucolor=black,
urlcolor=cyan,
breaklinks=true,
bookmarks=true,
bookmarksopen=false,
bookmarksnumbered=false,
pdfhighlight=/O,
]{hyperref}

%%%%%%%%%%%%%%%%%%%%%%%%%%%%%%%%%%%%%%%%%%%%%%%%%%%%%%%%%%%%%
%% FANCY HEADERS
%%%%%%%%%%%%%%%%%%%%%%%%%%%%%%%%%%%%%%%%%%%%%%%%%%%%%%%%%%%%%
% --- Kopf- und Fusszeilen - {} = rechts (gerade), [] = links (ungerade)
% letzte seite: \pageref{LastPage}
% doppelseitig:
%\lhead{Elektronik: \textbf{Oszilatorschaltungen}}	\chead{}		\rhead{Cyril Stoller und Hannes Stauffer}
%\lfoot{\today}	\cfoot{}		\rfoot{Seite \thepage\ von \pageref{LastPage}}

% einseitig:
%\lhead{\rightmark}			\chead{}					\rhead{}
%\lfoot{\leftmark}			\cfoot{}					\rfoot{Seite \thepage\ von \pageref{\LastPage}}

%\setlength{\headrulewidth}{0.4pt}
%\setlength{\footrulewidth}{0.4pt}


% Formeln r�misch nummerieren
\renewcommand{\theequation}{\Roman{equation}} 

% "Formel" statt "Gleichung"
\def\equationname{Formel}

%%%%%%%%%%%%%%%%%%%%%%%%%%%%%%%%%%%%%%%%%%%%%%%%%%%%%%%%%%%%%
%% DOKUMENT
%%%%%%%%%%%%%%%%%%%%%%%%%%%%%%%%%%%%%%%%%%%%%%%%%%%%%%%%%%%%%
\begin{document}

\title{Projektarbeit: High Power RGB-LED}
\date{\today}
\author{Cyril Stoller, Marcel B�rtschi}
\maketitle

%% Inhaltsverzeichnis %%%%%%%%%%%%%%%%%%%%%%%%%%%%%%%%%%%%%%%
\tableofcontents %Inhaltsverzeichnis

\vfill

\listoffigures

%\pagestyle{fancy} %%Ab hier die Kopf-/Fusszeilen: headings / fancy / ...

\newpage

\begin{abstract}
	
\begin{center}	
\textbf{Abstract}
\vspace{0.3cm}

	Um unsere Kentnisse in VHDL und im Entwerfen von elektronischen Schaltungen zu vertiefen, haben wir ein Projekt erarbeitet, wo wir beides gleichermassen �ben konnten. Dabei ist das Projekt 40 Watt high power LED entstanden. 
\end{center}
	
\end{abstract}

\vspace{2cm}


%%%%%%%%%%%%%%%%%%%%%%%%%%%%%%%%%%%%%%%%%%%%%%%%%%%%%%%%%%%%%
%%                                                         %%
%%         Kapitel / Hauptteil des Dokumentes              %%
%%                                                         %%
%%%%%%%%%%%%%%%%%%%%%%%%%%%%%%%%%%%%%%%%%%%%%%%%%%%%%%%%%%%%%



\section{Ziel}
\begin{quote}
Es gibt nichts Praktischeres als eine gute Theorie.

Immanuel Kant (1724 - 1804), deutscher Philosoph
\end{quote}
Um trozem mal selbst handanzulegen, ist ein Projekt zu erarbeiten, dass einen digitalen und einen analogen Teil enth�lt. Der Zeitliche Rahmen ist auf ein halbes Semester begrenzt. 
Der digitale Teil soll mit mit einem Spartan 3E Board in VHDL realisiert werden und f�r den analogen Teil sind keine Vorgaben gemacht.



\section{Schlussfolgerung}


\vfill
\begin{tabular}{rr}
	\\
	\\
	\\
	\\
	\toprule
	\scriptsize{Datum und Unterschrift}	\hspace{3cm}	&	\textsc{Marcel B�rtschi}	\\
	\\
	\\
	\\
	\\
	\toprule
	\scriptsize{Datum und Unterschrift}	\hspace{3cm}	&	\textsc{Cyril Stoller}
\end{tabular}


% Der Anhang kommt auf eine neue Zeile
\newpage
% Offizielle "A Anhang" Aufz�hlungsvariante
\appendix
% Nur im Inhaltsverzeichnis hinzuf�gen (mit richtiger Seite, da vorher "\newpage"), aber kein Text
\addcontentsline{toc}{section}{Anhang}

% Quellenverzeichnis
%\addcontentsline{toc}{section}{Quellenverzeichnis}
\section{Quellenverzeichnis}
\renewcommand\refname{}

\vspace{-1cm}

\bibliographystyle{amsplain}
\bibliography{Bildquellen}

\end{document}
