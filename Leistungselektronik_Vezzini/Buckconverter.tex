%% Dokumenteinstellungen %%%%%%%%%%%%%%%%%%%%%%%%%%%%%%%%%%%%
\documentclass[a4paper,oneside,12pt,ngerman]{scrartcl}

%% Deutsche Anpassungen %%%%%%%%%%%%%%%%%%%%%%%%%%%%%%%%%%%%%
\usepackage[ngerman]{babel}
\usepackage[T1]{fontenc}
\usepackage[ansinew]{inputenc}
\usepackage{lmodern} %Type1-Schriftart f�r nicht-englische Texte
\usepackage{booktabs}	% sch�nere tabellen

%% Packages f�r Grafiken & Abbildungen %%%%%%%%%%%%%%%%%%%%%%
\usepackage{graphicx} %%Zum Laden von Grafiken
%\usepackage{subfig} %%Teilabbildungen in einer Abbildung
%\usepackage{tikz} %%Vektorgrafiken aus LaTeX heraus erstellen


%% Packages f�r Formeln %%%%%%%%%%%%%%%%%%%%%%%%%%%%%%%%%%%%%
\usepackage{amsmath}
\usepackage{amsthm}
\usepackage{amsfonts}


%% Andere Packages %%%%%%%%%%%%%%%%%%%%%%%%%%%%%%%%%%%%%%%%%%
%\usepackage{a4wide} %%Kleinere Seitenr�nder = mehr Text pro Zeile.
\usepackage{fancyhdr} %%Fancy Kopf- und Fu�zeilen
%\usepackage{longtable} %%F�r Tabellen, die eine Seite �berschreiten
\usepackage{lastpage}
\usepackage[raggedright]{subfigure}
\usepackage[final]{pdfpages}
\includepdfset{pages=-,noautoscale}

%%%%%%%%%%%%%%%%%%%%%%%%%%%%%%%%%%%%%%%%%%%%%%%%%%%%%%%%%%%%%
%% TODO
%%%%%%%%%%%%%%%%%%%%%%%%%%%%%%%%%%%%%%%%%%%%%%%%%%%%%%%%%%%%%
% 
% 
%%%%%%%%%%%%%%%%%%%%%%%%%%%%%%%%%%%%%%%%%%%%%%%%%%%%%%%%%%%%%



%%%%%%%%%%%%%%%%%%%%%%%%%%%%%%%%%%%%%%%%%%%%%%%%%%%%%%%%%%%%%
%% Optionen / Modifikationen
%%%%%%%%%%%%%%%%%%%%%%%%%%%%%%%%%%%%%%%%%%%%%%%%%%%%%%%%%%%%%
%%%%%%%%%%%%%%%%%%%%%%%%%%%%%%%%%%%%%%%%%%%%%%%%%%%%%%%%%%%%%
%%                                                         %%
%%                     EINSTELLUNGEN                       %%
%%                                                         %%
%%%%%%%%%%%%%%%%%%%%%%%%%%%%%%%%%%%%%%%%%%%%%%%%%%%%%%%%%%%%%

%%%%%%%%%%%%%%%%%%%%%%%%%%%%%%%%%%%%%%%%%%%%%%%%%%%%%%%%%%%%%
%% HYPER REF
%%%%%%%%%%%%%%%%%%%%%%%%%%%%%%%%%%%%%%%%%%%%%%%%%%%%%%%%%%%%%
\usepackage[
hyperindex=true,
colorlinks=true,
linkcolor=black,
citecolor=black,
filecolor=black,
menucolor=black,
urlcolor=cyan,
breaklinks=true,
bookmarks=true,
bookmarksopen=false,
bookmarksnumbered=false,
pdfhighlight=/O,
]{hyperref}

%%%%%%%%%%%%%%%%%%%%%%%%%%%%%%%%%%%%%%%%%%%%%%%%%%%%%%%%%%%%%
%% FANCY HEADERS
%%%%%%%%%%%%%%%%%%%%%%%%%%%%%%%%%%%%%%%%%%%%%%%%%%%%%%%%%%%%%
% --- Kopf- und Fusszeilen - {} = rechts (gerade), [] = links (ungerade)
% letzte seite: \pageref{LastPage}
% doppelseitig:
%\lhead{Elektronik: \textbf{Oszilatorschaltungen}}	\chead{}		\rhead{Cyril Stoller und Hannes Stauffer}
%\lfoot{\today}	\cfoot{}		\rfoot{Seite \thepage\ von \pageref{LastPage}}

% einseitig:
%\lhead{\rightmark}			\chead{}					\rhead{}
%\lfoot{\leftmark}			\cfoot{}					\rfoot{Seite \thepage\ von \pageref{\LastPage}}

%\setlength{\headrulewidth}{0.4pt}
%\setlength{\footrulewidth}{0.4pt}


% Formeln r�misch nummerieren
\renewcommand{\theequation}{\Roman{equation}} 

% "Formel" statt "Gleichung"
\def\equationname{Formel}

%%%%%%%%%%%%%%%%%%%%%%%%%%%%%%%%%%%%%%%%%%%%%%%%%%%%%%%%%%%%%
%% DOKUMENT
%%%%%%%%%%%%%%%%%%%%%%%%%%%%%%%%%%%%%%%%%%%%%%%%%%%%%%%%%%%%%
\begin{document}

\title{Projektarbeit: Buckconverter (Step-Down)}
\date{\today}
\author{Cyril Stoller, Jascha Haldemann, Marcel B�rtschi, Nicola K�ser}
\maketitle


%\pagestyle{fancy} %%Ab hier die Kopf-/Fusszeilen: headings / fancy / ...

\vspace{1cm}


%%%%%%%%%%%%%%%%%%%%%%%%%%%%%%%%%%%%%%%%%%%%%%%%%%%%%%%%%%%%%
%%                                                         %%
%%         Kapitel / Hauptteil des Dokumentes              %%
%%                                                         %%
%%%%%%%%%%%%%%%%%%%%%%%%%%%%%%%%%%%%%%%%%%%%%%%%%%%%%%%%%%%%%

\section{Ziel}
Mithilfe der im Unterricht erarbeiteten Theorie und der Application Note von Infineon sollen alle Berechnungen f�r die Verluste durchgef�hrt werden und eine Drossel und ein Kondensator dimmensioniert werden. 

\section{Schema}
Ein Buckconverter oder Abw�rtswandler funktioniert nach folgendem Schema. Die PWM-Ansteuerung des MOSFET wurde hier einfachhietshalber weggelassen.
\begin{figure}[ht]
	\centering
		\includegraphics[width=0.90\textwidth]{pics/schema.pdf}
	\caption{Schema Stepdown Converter}
	\label{fig:schema}
\end{figure}

\section{Wirkungsgrad}
Um den Wirkungsgrad zu bestimmen muss zuerst die gesamte Verlustleistung ausgerechnet werden. Dies enth�lt die Schalt-und Leitverluste. Die Leckverluste werden vernachl�ssigt. 
\subsection{Leitverluste}
Die Leitverluste werden mit einfachen Modellen der Bauteile berechnet. Dabei wird z.B. beim MOSFET nur der Drain-Source Widerstand bestimmt und mit dem Quadrat des Effektivwerts des durchfliessenden Stromes multipliziert. 
\subsection{Schaltverluste}


\section{Dimensionierung Drossel}

\section{Dimensionierung Kondensator}

% Der Anhang kommt auf eine neue Zeile
\newpage
% Offizielle "A Anhang" Aufz�hlungsvariante
\appendix
% Nur im Inhaltsverzeichnis hinzuf�gen (mit richtiger Seite, da vorher "\newpage"), aber kein Text
\addcontentsline{toc}{section}{Anhang}

% Quellenverzeichnis
%\addcontentsline{toc}{section}{Quellenverzeichnis}
\section{Quellenverzeichnis}
\renewcommand\refname{}

\vspace{-1cm}

\bibliographystyle{amsplain}
\bibliography{Bildquellen}

\end{document}
